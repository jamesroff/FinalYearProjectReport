\begin{figure}[!ht]
    \centering
    % \includegraphics[width=0.7\textwidth]{Figures/QWMSimulation/SustOscilWithImpact/ValvePosition.png}
    % \includegraphics[width=0.7\textwidth]{Figures/QWMSimulation/SustOscilWithImpact/Velocity.png}
    % \includegraphics[width=0.7\textwidth]{Figures/QWMSimulation/SustOscilWithImpact/B.png}
    % \includegraphics[width=0.7\textwidth]{Figures/QWMSimulation/SustOscilWithImpact/C.png}
    \includegraphics[width=0.7\textwidth]{Figures/QWMSimulation/SustOscilWithImpact/CombinedFigure.png}
    % \caption{Simulation of QW with $\gamma = 14.9501$, $q = 0.6$, $\Lambda = 0$, $\alpha = 8.5658$, $\delta = 1$, $\kappa = 0$, $\beta = 0$, $\mu = 0.1407$, $\sigma = 10.3808$, $\phi = 0$ and $r = 0.8$. Equilibrium pressure is $p = 0.1686$ but the tank is actually held at $p = 0.0351$. Pressure comes from close to calculated instability boundary.}
    \caption{Simulation of the fixed tank pressure quarter wave model, \cref{eq: QWFixedTankPressure}, for $\gamma = 14.9501$ and $p_0 = 0.0351$. This can be seen as a red circle [\textcolor{Red}{o}] in \cref{fig: BifurcationDiagram}. The other parameters are as in \cref{tab: ValveClosingQWMParameterValues}, except $\Lambda=0$ and $\phi=0$.}
    \label{fig: QWSustOsc}
\end{figure}