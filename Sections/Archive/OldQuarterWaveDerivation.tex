% At the inlet of the pipe, an ideal inflow from the tank using Bernoulli's equation allows the first boundary condition

% \begin{equation*}
%     p_t(t) = p(0,t) + \frac{1}{2} \rho(0,t) v(0,t)^2
% \end{equation*}

% Mass conservation at the outlet of the pipe gives the second boundary condition of

% \begin{equation*}
%     \dot{m}(L,t) = \rho(L,t) A_p v(L,t) = \dot{m}_d(t)
% \end{equation*}

The assumed velocity and pressure distributions are a quarter-wave of the inlet pipe, given by
~
\begin{equation} %\label{eq: PressVelDist}
\begin{split}
    p(\eta,t) &= p_t(t) + B(t) \fun{sin}{2\pi \frac{\eta}{4L}} \, ,\\
    v(\eta,t) &= v_L(t) + C(t) \fun{cos}{2\pi \frac{\eta}{4L}} \, .
\end{split}
\end{equation}

These distributions fulfil the boundary conditions of equalling the tank pressure ($p_t(t)$) at $\eta = 0$ and mass conservation with the discharge at $\eta = L$. Here, the velocity at $\eta = L$ is
~
\begin{equation*}
    v_L(t) = \frac{\sqrt{2} \pi D C_d}{A_p \sqrt{\rho}} x(t) \sqrt{p_t(t)+B(t)} \, .
\end{equation*}

Substituting the velocity and pressure distributions given by \cref{eq: PressVelDist} into the PDE describing the fluid flow (\cref{eq: FluidPDE}) gives a differential equation which must be satisfied at all locations along the pipe. Using the collocation technique, it is sufficient that this differential equation is only satisfied at the mid-point of the pipe, where $\eta = \frac{L}{2}$. Here, $\fun{sin}{2 \pi \frac{\eta}{4L}} = \fun{cos}{2 \pi \frac{\eta}{4L}} = \frac{1}{\sqrt{2}}$. This requires that
~
\begin{equation*}
\begin{split}
    \sqrt{2} \dot{p}_t(t) + \dot{B}(t) + \frac{\pi \sqrt{2}}{4 L} B(t) \left( \sqrt{2} v_L(t) + C(t) \right) &= a^2 \rho \frac{\pi}{2L} C(t) \, , \\
    \sqrt{2} \dot{v}_L(t) + \dot{C}(t) - \frac{\pi \sqrt{2}}{4 L} C(t) \left( \sqrt{2} v_L(t) + C(t) \right) &= - \frac{1}{\rho} \frac{\pi \sqrt{2}}{2 L} B(t) \\
    &\qquad + \lambda \frac{L}{2D} \left( \sqrt{2} v_L(t) + C(t) \right) \left| \sqrt{2} v_L(t) + C(t) \right| \, .
\end{split}
\end{equation*}

Note that these two equations form a set of differential equations for $\dot{B}(t)$ and $\dot{C}(t)$, which can be added to the original three first order differential equations describing the motion of the valve and pressure in the tank. Also note the last term including $\lambda$ is a frictional loss through the pipe, which may initially be neglected. Also, $\dot{v}_L(t)$ is used instead of the lengthy expression, which is
~
\begin{equation*}
    \dot{v}_L(t) = \frac{\sqrt{2} \pi D C_d}{A_p \sqrt{\rho}} \sqrt{p_t(t) + B(t)} \left( \dot{x}(t) + \frac{1}{2} x(t) \frac{\dot{p}_t(t) + \dot{B}(t)}{p_t(t) + B(t)} \right) \, .
\end{equation*}

The differential equation describing the tank pressure, using the assumed velocity distribution, can be written as
~
\begin{equation*}
    \dot{p}_t(t) = \frac{a^2}{V} \left( \dot{m}_{in} - \rho A_p C(t) - \sqrt{2} \pi D C_d x(t) \sqrt{\rho} \sqrt{p_t(t) + B(t)} \right) \, .
\end{equation*}

The full set of differential equations are
~
\begin{equation*}
\begin{split}
    \dot{x} &= v \\
    \dot{v} &= - \frac{c_v}{m_v} v + \frac{A_p}{m_v} B + \frac{\zeta^2}{m_v A_p} x^2 B + \frac{\zeta^2}{m_v A_p} x^2 p_t - \frac{A_v - A_p}{m_v} p_t \\
    \dot{p}_t &= \frac{a^2}{V} \left( \dot{m}_{in} - \rho A_p C(t) - \sqrt{2} \pi D C_d x(t) \sqrt{\rho} \sqrt{p_t(t) + B(t)} \right) \, , \\
    \dot{B} &= a^2 \rho \frac{\pi}{2L} C - \sqrt{2} \dot{p}_t - \frac{\pi}{2L} \frac{\sqrt{2} \pi D C_d}{A_p \sqrt{\rho}} x B \sqrt{p_t + B} - \frac{\pi \sqrt{2}}{4L} B C \, , \\
    ~
    \dot{C} &=
    \frac{\pi \zeta}{2 A_p L \sqrt{\rho}} x C \sqrt{p_t + B}
    + \frac{\pi \sqrt{2}}{4L} C^2
    - \frac{\pi}{2 L \rho} B
    - \frac{\sqrt{2} \zeta}{A_p \sqrt{\rho}} v \sqrt{p_t + B}
    + \frac{2 - \sqrt{2}}{2} \frac{\zeta}{A_p \sqrt{\rho}} \frac{x}{\sqrt{p_t + B}} \dot{p}_t \\ &\quad %linebreak
    - \frac{\sqrt{2}\pi\zeta a^2 \rho}{4 L A_p \sqrt{\rho}} \frac{x C}{\sqrt{p_t + B}}
    + \frac{\sqrt{2}\pi\zeta^2}{4L A_p\,^2 \rho} x^2 B
    + \frac{\pi \zeta}{4 L A_p \sqrt{\rho}} \frac{x B C}{\sqrt{p_t + B}}
    + \lambda \frac{\sqrt{2}L}{4 D} \left( \sqrt{2} v_L + C \right) \left| \sqrt{2} v_L + C \right|
\end{split}
\end{equation*}

where $\zeta = \sqrt{2} \pi D C_d$.