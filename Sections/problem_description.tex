\section{Introduction}

Pressure Relief Valves (PRVs) are an important safety feature in many pressurised systems, forming a final defence against over pressurisation. A fluid operating at a higher pressure than designed can lead to decreases in efficiency for some processes. More severely, a fluid within a vessel reaching high pressures can cause an explosive release of fluid. This has two serious impacts. Firstly, severe damage can cause costly repairs possibly including replacing the entire pressure tank. There is not only the capital cost of repairs, but also the potential loss of revenue for being unable to use a process. Secondly, it could prove fatal to an unfortunate operator.

The simplest type of PRV is the direct spring-operated PRV. These are widely used, and rely on a spring to hold a poppet in place. As the pressure increases, a greater force acts upon the poppet to open the valve. Hence, the poppet will rise which allows fluid to escape causing the pressure in the tank to be relieved. PRVs are widely used as they prevent unnecessary waste of fluid because the valve will close once the pressure has decreased. However, spring-operated PRVs do not scale well for larger vessels as larger flow rates are needed requiring a very large physical space.

A more complex design of PRV is the Pilot-Operated PRV (see \cref{fig: Diagram} below). This allows fluid both above and below the piston, and so uses the fluid pressure to keep the valve shut. When the pressure becomes too great, a small pilot valve opens which reduces the pressure above the piston. Hence, the main valve opens allowing the fluid to escape and relieve the pressure in the vessel. This has several advantages. Even for large tanks, no physical spring is needed, so the valve can be physically smaller than a spring-operated PRV.

Unfortunately, both spring and pilot operated PRVs do not always behave as expected. Several forms of instability exist, ranging from a desirable blow-down effect to the most damaging chatter instability~\cite{Hos2017DynamicRecommendations}. A blow-down effect is when the PRV closes at a lower pressure than that at which it opens, called the set pressure. This is desirable as it can help the valve remain slightly open rather than cycling, where the valve intermittently opens and closes.

The more damaging chatter instability is characterised by high frequency oscillations in which the poppet or piston undergoes impacts. These oscillations contain large amounts of energy, which can cause severe damage to the PRV. This damage can decrease the performance of the valve, or even cause it to fail completely. These consequences have serious repercussions for the pressurised vessel as the final safety precaution is compromised, meaning the dangerous effects of over pressurisation may occur.

This project aims to study the various types of instability which can occur for a pilot-operated PRV. Particular focus is placed on identifying the damaging chatter instability. If successful in establishing a stability criteria, this work will allow a PRV to be designed to avoid chatter. This report will begin by reviewing the current understanding surrounding instability in PRVs.

%what are you working on?
%what is the real-world challenge?
%why should anyone care?