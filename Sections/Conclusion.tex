\chapter{Conclusion}

The research presented here has successfully developed a reduced-order dynamical model of a pilot-operated pressure relief valve immediately after the pilot valve has closing and the main valve begins to close. The model was derived from first principles, and a model reduction was performed using a collocation technique. An fixed tank pressure analysis was performed and identified the point at which the main valve piston acts as a negative damper on the quarter-wave within the inlet piping, where a Hopf bifurcation was expected to occur. This analysis was compared to a numerical continuation of the Hopf bifurcation found.

Numerical simulations of both the fixed tank pressure and full quarter-wave models were performed, which successively demonstrated that the model can reproduce the flutter and chatter instabilities. These simulations revealed a progression from stable operation to flutter to chatter as the pipe length is increased. For low pipe lengths, the main piston closes with any oscillations decaying. For longer pipe lengths, an apparent flutter instability occurs, where oscillations grow in time but the main piston closes sufficiently quickly that chatter does not occur. For sufficiently long pipe lengths, chatter occurs where the oscillations grow faster than the valve closes. Although further analysis must be performed to investigate the stability boundary for chatter to occur, an analytical boundary was found at which chatter occurs.