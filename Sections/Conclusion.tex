\chapter{Conclusion}

A reduced-order dynamical model of a pilot-operated pressure relief valve (PRV) closing has been successfully developed, valid immediately after the pilot valve has closed and the main valve begins closing. The model was derived from first principles, and a model reduction was performed using a collocation technique. For many scenarios considered, the pilot operated PRV closes, demonstrating why a pilot operated PRV may be preferable to the simpler direct spring-operated PRV. Nevertheless, evidence of oscillatory behaviour suggested by Allison and Brun \cite{Allison2015TestingValves} is found. Fixed tank pressure analysis identifies the point at which the main valve piston acts as a negative damper on the quarter-wave within the inlet piping, initially believed to correspond to a Hopf bifurcation. A comparison to a numerical continuation of the Hopf bifurcation is then performed.

Numerical simulations of both the fixed and variable tank pressure quarter-wave models are then performed. These successively demonstrate the models can reproduce both flutter and chatter instabilities. A progression from stable operation to flutter to chatter is observed as the pipe length is increased. For low pipe lengths, the main piston closes with any oscillations decaying. For longer pipe lengths, an apparent flutter instability occurs, where oscillations grow in time but the main piston still closes. For excessively long pipe lengths, chatter occurs where the oscillations grow faster than the valve closes.

Many opportunities for further analysis into the instabilities of pilot-operated PRVs are identified. To greater understand how damaging chatter instabilities are, a more complex model including the pilot valve is required. Perhaps the oscillations can cause the pilot valve to open, causing a more extreme instability. Nonetheless, an analytical stability boundary beyond which chatter occurs has been found. However, this stability boundary must be validated against both CFD and experimental results. Finally, some differences are identified between the direct spring-operated and pilot-operated PRVs. One important difference, where pilot-operated PRVs develop stable flutter oscillations before developing into chatter, may help to explain why chatter is less common for pilot-operated PRVs.