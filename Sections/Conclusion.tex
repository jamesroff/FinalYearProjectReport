\chapter{Conclusion}

The research presented successfully developed a reduced-order dynamical model of a pilot-operated pressure relief valve closing, valid immediately after the pilot valve has closed and the main valve begins closing. The model was derived from first principles, and a model reduction was performed using a Collocation technique. A fixed tank pressure analysis identified the point at which the main valve piston acts as a negative damper on the quarter-wave within the inlet piping, corresponding to a Hopf bifurcation. A comparison to a numerical continuation of the Hopf bifurcation was then performed.

Numerical simulations of both the fixed tank pressure and variable tank pressure quarter-wave models were performed. These successively demonstrated the models can reproduce both flutter and chatter instabilities. A progression from stable operation to flutter to chatter was observed as the pipe length is increased. For low pipe lengths, the main piston closes with any oscillations decaying. For longer pipe lengths, an apparent flutter instability occurs, where oscillations grow in time but the main piston closes more quickly. For excessively long pipe lengths, chatter occurs where the oscillations grow faster than the valve closes.

Many opportunities for further analysis into the instabilities of pilot-operated PRVs have been identified. Nonetheless, an analytical stability boundary beyond which chatter occurs has been found. However, the stability boundary must be validated against both CFD and experimental results. Finally, some differences are identified between the direct spring-operated and pilot-operated PRVs. One important difference, where pilot-operated PRVs develop stable flutter oscillations before developing into chatter, may help to explain why chatter is less common for pilot-operated PRVs.