\chapter{Further work}

Unlike for the direct spring-operated PRV, little research has been performed on the more complex pilot-operated PRV, leaving many opportunities for further work to be carried out. Most existing research focuses on experimental results and simple analytical models to reproduce particular behaviours seen during these experimental results \cite{Botros1997Riser-ReliefInteractions,Zung2002NonlinearDesigners,Ye2009DynamicSystem,Allison2015TestingValves}. However, currently no computational fluid dynamic (CFD) models have been used to study pilot-operated PRVs.

Further work should aim to validate the quarter-wave model, using both CFD and experimental data. As the unstable valve open equilibrium has not previously been observed, validation of the models proposed here is important.
Validation of an analytical model against real world data gives a further verification of a models accuracy than CFD models. Additionally, a prediction of the chatter instability boundary can be used for pilot-operated PRV design, the prediction must be shown to be realistic using experimental results.
% Further work using CFD models will be useful for many reasons. One important reason is it allows a method to validate analytical models such as the quarter-wave model proposed here against complex fluid models far more quickly than using experimental data. This is particularly important for both models presented in this work, as the valve open equilibrium being unstable is a phenomena which has not been observed previously. Additionally, if an analytical prediction of a stability boundary such as in \cref{fig: BifurcationDiagram} is proposed, multiple CFD simulations can be performed in both the stable and unstable regions to valid the stability prediction.

The quarter wave model presented in \cref{sec: QWM} is still rich for further exploration as an interesting nonlinear dynamical system. A Hopf bifurcation of an unstable equilibrium displays some very interesting behaviour, and the work so far performed was unable to identify the limit cycle involved. However, the main unexplained feature of the simulations is the difference in frequency between the quarter-wave and the oscillations observed.

An important finding is the possibility of a transition from flutter and chatter. \Cref{fig: UnstableHopf(Long)} suggests prolonged flutter may occur for a pilot-operated PRV, unlike for a spring-operated PRV which will quickly develop into chatter~\cite{Hos2016DynamicService}. Hence, the transition between the dynamics shown in \cref{fig: UnstableHopf(Long)} and \cref{fig: QWSustOsc} may represent a boundary between a flutter instability and the far more damaging chatter instability.

The final aspect requiring further exploration is the varying tank pressure quarter-wave model, \cref{eq: QWFixedTankPressure}. \Cref{fig: QWNearEquil} suggests that the fixed tank pressure assumption is valid for approximating the stability boundary, as in \cref{subsec: QWMAnalyticalBound}. However, similar numerical continuation techniques to those seen in \cref{sec: Continuation} can calculate a similar stability boundary of a Hopf bifurcation for the varying tank pressure model. %More interesting may be comparing the transition of flutter to chatter for a fixed tank pressure and varying tank pressure.

One major area for further work is including the pilot valve into the quarter-wave model. An even greater range of instabilities can be studied, including the common cycling instability. Cycling cannot be reproduced in \cref{eq: FullQWMDimensionless} as the main valve has no opening mechanism. Including the pilot valve would reveal if the main piston is below the unstable equilibrium when the pilot valve closes, such that the main valve will close. Finally, if the pilot valve affects the flutter and chatter instability may be studied, although the previous experimental results suggests otherwise~\cite{Allison2015TestingValves}.
% 
% Although existing work has performed many different experiments, further experimental results are still required for two reasons. Firstly, they allow a validation of any analytical model against real world data, giving a higher validation that the model is realistic than comparison to CFD models. Secondly, before any analytical prediction of a chatter instability boundary can be used during the design of a pilot-operated PRV, it must be shown to be a reasonable approximation for the real world.
% 
% % The research presented in this work has revealed several key differences between models for a spring-operated and pilot-operated PRVs. However, there are significant areas revealed by these models which need greater exploration in further work.

% Firstly, the quarter-wave model presented which shows many of the unstable behaviours of a pilot-operated PRV must be validated as a model. Significantly more computational fluid dynamics (CFD) models and experimental results are needed to aid in the modelling process of the PRV. One key finding of this report is that the main piston closes because the valve-open equilibrium is unstable (\cref{sec: Prog}).

\newpage
\begin{itemize}
    \item More CFD and experimental results required:
    \begin{itemize}
        \item Validate the quarter-wave model
        \item Is the valve open equilibrium unstable? Can the valve open further?
        \item Identify flutter and chatter behaviour for given parameters
    \end{itemize}
    \item Further exploration of this analytical model
    \begin{itemize}
        \item Why are the oscillations not at the quarter-wave frequency?
        \item Does the stability boundary change by allowing the tank pressure to vary?
    \end{itemize}
    \item Extension this analytical model
    \begin{itemize}
        \item Add pilot valve dynamics - allows 
        \item Immediately after pilot valve closes, is the initial conditions below the unstable equilibrium?
        \item Does the model present other instabilities such as cycling?
        \item How does the pilot valve effect the flutter and chatter instability?
    \end{itemize}
    \item Experimental results
        \begin{itemize}
        \item Validate the quarter-wave model
        \item Is the valve open equilibrium unstable? Can the valve open further?
        \item Offer opportunity to compare calculated stability boundaries against "true" data
    \end{itemize}
\end{itemize}