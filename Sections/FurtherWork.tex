\chapter{Further work}

Unlike for the direct spring-operated PRV, little research has been performed on the more complex pilot-operated PRV, leaving many opportunities for further work to be carried out. Most existing research focuses on experimental results and simple analytical models to reproduce particular behaviours seen during these experimental results \cite{Botros1997Riser-ReliefInteractions,Zung2002NonlinearDesigners,Ye2009DynamicSystem,Allison2015TestingValves}. However, currently no computational fluid dynamic (CFD) models have been used to study pilot-operated PRVs.

Further work using CFD models will be useful for many reasons. One important reason is it allows a method to validate analytical models such as the quarter-wave model proposed here against complex fluid models far more quickly than using experimental data. This is particularly important for both models presented in this work, as the valve open equilibrium being unstable is a phenomena which has not been observed previously. Additionally, if an analytical prediction of a stability boundary such as in \cref{fig: BifurcationDiagram} is proposed, multiple CFD simulations can be performed in both the stable and unstable regions to valid the stability prediction.

Additionally, the quarter wave model presented in \cref{sec: QWM} is still rich for further exploration as an interesting nonlinear dynamical system. A Hopf bifurcation of an unstable equilibrium displays some very interesting behaviour, and the work so far performed was unable to identify the limit cycle involved. However, the main unexplained feature of the simulations performed is any oscillations do not appear to occur at the fundamental quarter-wave frequency which is the only mode included in the model.

A important aspect for studying flutter and chatter is the transition between them. \Cref{fig: UnstableHopf(Long)} suggests that it may be possible for prolonged flutter to occur for a pilot-operated PRV, unlike for a spring-operated PRV which will quickly develop into chatter~\cite{Hos2016DynamicService}. Hence, where the transition between the dynamics shown in \cref{fig: UnstableHopf(Long)} and \cref{fig: QWSustOsc} occurs may represent a boundary between a flutter instability and the far more damaging chatter instability.

The final aspect of the quarter-wave model presented in \cref{sec: QWM} which requires further exploration is the effect of a varying tank pressure. \Cref{fig: QWNearEquil} suggests that relaxing the fixed tank pressure assumption made in \cref{subsec: QWMAnalyticalBound} still provides a suitable approximation of the stability boundary. However, the effect of relaxing this assumption can easily be explored using similar numerical continuation techniques seen in \cref{sec: Continuation}. More interesting may be comparing the transition of flutter to chatter for a fixed tank pressure and varying tank pressure.

One significant piece of further work would be to include the pilot valve into the quarter-wave model. This should allow an even greater range of instabilities to occur, including the common cycling instability which cannot be reproduced in \cref{eq: FullQWMDimensionless} as the main valve can never re-open. Including the pilot valve should also allow greater study on the closing dynamics, and would reveal if when the pilot valve closes, the main piston is below the unstable equilibrium so will also close. Finally, the effect of the pilot valve on the flutter and chatter instability may be studied, although the previous experimental results suggests the pilot valve will not open~\cite{Allison2015TestingValves}.

Although existing work has performed many different experiments, further experimental results are still required for two reasons. Firstly, they allow a validation of any analytical model against real world data, giving a higher validation that the model is realistic than comparison to CFD models. Secondly, before any analytical prediction of a chatter instability boundary can be used during the design of a pilot-operated PRV, it must be shown to be a reasonable approximation for the real world.
% 
% % The research presented in this work has revealed several key differences between models for a spring-operated and pilot-operated PRVs. However, there are significant areas revealed by these models which need greater exploration in further work.

% Firstly, the quarter-wave model presented which shows many of the unstable behaviours of a pilot-operated PRV must be validated as a model. Significantly more computational fluid dynamics (CFD) models and experimental results are needed to aid in the modelling process of the PRV. One key finding of this report is that the main piston closes because the valve-open equilibrium is unstable (\cref{sec: Prog}).

\newpage
\begin{itemize}
    \item More CFD and experimental results required:
    \begin{itemize}
        \item Validate the quarter-wave model
        \item Is the valve open equilibrium unstable? Can the valve open further?
        \item Identify flutter and chatter behaviour for given parameters
    \end{itemize}
    \item Further exploration of this analytical model
    \begin{itemize}
        \item Why are the oscillations not at the quarter-wave frequency?
        \item Does the stability boundary change by allowing the tank pressure to vary?
    \end{itemize}
    \item Extension this analytical model
    \begin{itemize}
        \item Add pilot valve dynamics - allows 
        \item Immediately after pilot valve closes, is the initial conditions below the unstable equilibrium?
        \item Does the model present other instabilities such as cycling?
        \item How does the pilot valve effect the flutter and chatter instability?
    \end{itemize}
    \item Experimental results
        \begin{itemize}
        \item Validate the quarter-wave model
        \item Is the valve open equilibrium unstable? Can the valve open further?
        \item Offer opportunity to compare calculated stability boundaries against "true" data
    \end{itemize}
\end{itemize}